\documentclass[a4paper]{article}
\usepackage[english]{babel}
\usepackage{UserStories}
\usepackage{url}
\usepackage{listings}

\begin{document}
\author{Vitaliy Akimov}

\title{C++ generics library}

\newcommand{\gfun}[1]{\subsection{#1}}
\newcommand{\gfunname}[1]{\gfun{#1 function}}

\maketitle
\newpage

\tableofcontents
\newpage

\section{Desription}

Scrap Your Boilerplate (SYB) is approach for generic programming which is come from functional programming.
Actually, this file is a port of Haskell's library this the same name.

The main point of SYB approach is a generic traversal of data structure similar to well known in OOP world Visitor Pattern.
But traversal combinators from this library is more general so they are applicable in more areas.

See \url{http://www.cs.vu.nl/boilerplate/} for more information. Functions have similar names to Haskell's SYB library.

C++ template meta-programming technique is hardly used for implementation of this library.
Code style is similar to boost::mpl. And implemented metafunctions are compatible with it.
For template specialisation almost everywhere used \lstinline$boost::enable_if$ library.

Some changes are made due to imperativeness of C++ languege and imposibility to port some concepts from Haskell to C++.
Generic function is any functional object but its operator() should be template with one parameter.

The action is specialized by concrete type name, or by type's shape. The sum of products technique is used to encode types.
Any class or struct is product of its fields. Boost::variant is example of Sum type.
In OOP Sum type is Abstract class this some number of children, each is a different element of a Sum.
Although typical to OOP types is supported many boilerplate code programmer should write by himself.
But with for \lstinline$boost::variant$ great amount of this routine boilerplate job is done by this library.

In the document mostly functionality needed for License System project is described. But the library is applicable to any C++ project.

Not full set of C++ types are implemented at the moment. But implemented set is enough for large area of tasks.

Actually all set of type's primitives from OOP design could be defined here.

For first stage this library doesn't support heterogeneous members, but they can be easily emulated by boost::variant (Sum Type in FP world).

\section{Tasks}

\subsection{gfoldl function}

This function is implement's one level traversal with context passing.
Every other traversal combinator could be implemented on top of this function.
This is really the main function of the library.
Haskell implementation has another one \emph{gunfold} for generic data producers
but because of C++ nature we don't need it in the library.

\subsection{Generic producers functionality}

This is mostly policy for making generic producer function, e.g. parsers, random data generators, deserializations.

\subsection{Generic consumers functionality}

This is mostly policy for making generic consumer function, e.g. parsers, random data generators, serializations, queries.

\subsection{gmapT}
This function executes a generic function for each member of any object 
Only one direct sub-level is traversed.
The function is implemented by means of gfoldl but it's used more often.
\subsection{gmapTCopy}
Analogues to gmapQ but it copies all objects along with traversal, so parameter object is unchanged.
\subsection{gmapQ}

\subsection{everywhere}
Generic function which executes generic function passed to it as parameter for each sub member of specified object recursively.
It defers from gmapT which traverses only one level.
Actually the function is analogues to familiar OOP Visitor pattern.

\subsection{everywhereCopy}
Generic function which executes generic function passed to it as parameter for each sub member of specified object recursively.
It defers from gmapT which traverses only one level.
Actually the function is analogues to familiar OOP Visitor pattern.

\subsection{everywhereBut}

The function is similar to everwhere but it has additional predicate which asks the library to stop traversal.

\subsection{everything}

Generic function which executes generic function passed to it as parameter for each sub member of specified object recursively.

\subsection{Path combinators}

Expression for accessing some node in a heterogeneous try of objects.

\subsection{Type Aliases}

The facility is mostly needed for overloading. It gives ability to overload generic function more specifically.
I.e. type Salary is alias for float. And we need the alias to overload function only for Salary. And to keep rest of floats untouched.

\subsection{Command line parsing}

Ability to automatically derive command line parser for any data type with documentation.

\subsection{XML serialization}

Serialize any object to custom XML format.

\subsection{XML deserialization}

Deserialize any object from custom XML format.

\subsection{Axis2C serialization}

Serialize any object to a axis message.

\subsection{Axis2C deserialization}

Deserialize any object from a axis message.

\subsection{Generate Meta Data in form of ADT}

Generate ADT for data types for further processing with various specific tools, what are more supposed to deal with meta data processing.
E.g. term rewriting tools, Haskell.

\subsection{Generate documentation for data types}

This will be done by means of generated ADT. It will generate \LaTeX sources for farther processing.

\subsection{Java Types}

The library is used in heterogeneous systems, there are a few different languages and technologies are used.
The library will allow automatic generation of interface datatypes on Java Side.

\subsection{C wrappers}

We need C function wrapper to be available. Since all other language such as 4GL, Java, Haskell don't support FFI for C++. 
So we need automatic facility for exporting C++ to C.

\subsection{4GL wrappers}

The wrapper function which makes function available from 4GL code.

\subsection{ORM}

The function will automatically create needed schema, and all CRUD queries for the data type.

\subsection{UI} 
This is one more possible application of the library. We can automatically generate UI widgets.

This library provides facilities for automatic creation of GUI Widget from SYB-compatible type.
The most general way to create widget is:

\lstinline{QWidget *v = new GWidget<Type>();} \footnote{The description is taken from my previous implementation of a similar library 
(what was based on Qt)}

Representation and behaviour of a widget is based on a shape of a type.
But common specializations are defined for all widgets.
There are number of templates by means of which you can change default representation or behaviour.

A parent of a generic widget is always some Widget (depends on type shape).
For example for \lstinline$std::vector$ it could be TableView or TreeView or ListView.
The librery is hightly customizable by means of C++ template specializations.

Any generic widget must have parent with type \lstinline$StylelessWidget<T,...>$.
You could respecialize this template for changing behavioral aspects of a widget.

Widget's type depends on context of a widget. For example top level widget is usually \lstinline$GWidget<T>$.
But for widget embedded into compound form (product type) has type \lstinline$ProdItem<T,...>$.
So you can specialize changes of representation and behaviour by type and context.
Almost every template has additional parameter TraiT. It helps in more accurate template specializing.
The library defines 2 traits (Editable and Readonly). They show whether widget's content can be edited by a user or not.
Your own Trait should be inherited from one of these ones.

Every generic widget has methods:

\begin{lstlisting}
T get() const;			// usual setter method
void set(T const&);		// usual getter method

and signals:
void gChanged();		// this signals value of a widget is changed
void gEdited();			// this signals value of a widget is edited by a user
\end{lstlisting}

There are number of other features.

\begin{itemize}
\item Lazy updates of widget's value without blocking of UI (from remote server for example).
\item Binding value of a container's widget with it's element widget.
\item Validators.
\item ...
\end{itemize}

\subsection{Polymorphic types}

The facility is needed for encoding C++ polymorphic types. It's not a straightforward task because of the way C++ handles polymorphism. 
Here we use the technique from \lstinline$boost::serialization$.

\end{document}

