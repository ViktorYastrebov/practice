\documentclass[a4paper]{article}
\usepackage[english]{babel}
\usepackage{UserStories}
\usepackage{xcolor}
\usepackage{verbatim}
\usepackage{url}
\usepackage{listings}

\begin{document}

\lstset{backgroundcolor=\color{lightgray},language=xml, numbers=left, numbersep=0.3cm, frameround=fttt,columns=fullflexible,language=C++ %
  ,basicstyle=\tiny,commentstyle=\tiny}

\author{Vitaliy Akimov}

\title{C++ generics library}

\newcommand{\gfun}[1]{\subsection{#1}}
\newcommand{\gfunname}[1]{\gfun{#1 function}}

\maketitle
\newpage

\tableofcontents
\newpage

\section{Introduction}

Scrap Your Boilerplate (SYB) is approach for generic programming which is come from functional programming.
Actually, this file is a port of Haskell's library this the same name.

The main point of SYB approach is a generic traversal of data structure similar to well known in OOP world Visitor Pattern.
But traversal combinators from this library is more general so they are applicable in more areas.

See \url{http://www.cs.vu.nl/boilerplate/} for more information. Functions have similar names to Haskell's SYB library.

C++ template meta-programming technique is hardly used for implementation of this library.
Code style is similar to boost::mpl. And implemented metafunctions are compatible with it.
For template specialisation almost everywhere used \lstinline$boost::enable_if$ library.

Some changes are made due to imperativeness of C++ language and imposibility to port some concepts from Haskell to C++.

The action is specialized by concrete type name, or by type's shape.
Any class or struct is product of its fields. 

In the document mostly functionality needed for License System project is described. But the library is applicable to any C++ project.

Not full set of C++ types are implemented at the moment. But implemented set is enough for large area of tasks.

Actually all set of type's primitives from OOP design could be defined here.



\section{How to use generic functions}

\todo{clarify terminology (for example for Parent/Children is used both for fields of class and for %
descendant of polymorphic class)}

Generic function is usual C++ function which traverses data type tree and runs some action on each child. 
So it can be seen as a generic visitor. It more general than visitor. For example it can handle context
information, it can build object, and not only traverse and so on.

Data types of an object should be prepared to be traversable by generic function.

\subsection{Data types}

\subsubsection{Example}

The following example will be used for showing various aspects of the library:

\begin{lstlisting}

  struct Person {
    std::string &name() { return name_; }
		std::string &address() { return address_; }
	private:
 		std::string name_;
		std::string address_;
	};

	struct Employee {
		Person& person() {return person_; }
		float& salary() {return salary_; }
	private:
		Person person_;
		float salary_;
  };

  struct SubUnit{};

  struct PersonUnit:SubUnit{
		Employee& empl() { return empl_; }
	private:
		Employee empl_;
  };

	typedef Employee Manager;
  
	struct Dept {
		std::string& name() { return name_; }
	    Manager& manager() { return manager_; }
    std::list<boost::shared_ptr<SubUnit> >& subUnits() {
      return subUnits_;
    }
    std::string const& name() const { return name_; }
    Manager const& manager() const { return manager_; }
		std::list<boost::shared_ptr<SubUnit> > const& subUnits() const {
      return subUnits_;
    }
	private:
    std::string name_;
		Manager manager_;
		std::list<boost::shared_ptr<SubUnit> > subUnits_;
	};

	struct DeptUnit:SubUnit {
		Dept dept() { return dept_; }
	private:
		Dept dept_;
	};

  struct Company {
		std::list<Dept>& depts() { return depts_; }
	private:
		std::list<Dept> depts_;
	};
  
\end{lstlisting}

There are plenty of task we will have to do this the data types.

\begin{itemize}
  \item{Showing values in readable format for users}
  \item{Serializing/Deserializing values to/from binary/plain text/XML format}
  \item{Comparing values}
  \item{Mapping OOP values to RDBMS}
  \item{UI creation}
  \item{Command line arguments parsing}
  \item{Queries over values (similar to XPath queries)}
  \item{Transforming values}
\end{itemize}

The library can make everything (unfortunately most of it is not implemented yet).
The only thing you need to do for everything started working is provide description
of the data the library can understand.

Library itself is headers only. All you need to use it is just include Syb/SybAll.h.
Some functions (e.g. Generate) requires linking to the library.

There are a few preprocessor macroses for this (macroses are used only on top level
and the library doesn't depend on them). 

\subsubsection{Record types}

For a plain not polymorphic data type you need to provide list of fields the data
type contains. There are several ways for doing this and the easiest one is 
putting macros \lstinline{TRAVERSABLE} into your class definition in following way:

\begin{lstlisting}

struct Employee {
	Person& person() {return person_; }
	float& salary() {return salary_; }
private:
	Person person_;
	float salary_;

  TRAVERSABLE_BEGIN(Employee)
    TRAVERSABLE_FIELD(person_)
    TRAVERSABLE_FIELD(salary_)
  TRAVERSABLE_END
 };

\end{lstlisting}

It's better to put it just before closing brace of class definition. This will unify
the code style and will save us from a member's visibility troubles.

Note encapsulation is broken (like in any serialization library). But it's different 
paradigm. Logically it's not broken since any generic function anyway won't know 
any sensible information about members.

Note \lstinline{p.} in field list. The list will go to a friend function body, and 
p variable will be bound to object with Employee type.

That's all you need to make all generic functions working for the type.

\begin{comment}
The field list can be presented no only within class definition. It's preferable way 
but for classes we can not change because they are defined in some external library
or by any other reason. You can use \lstinline{TRAVERSABLE_OUTER} macros with the same
parameters it can be defined anywhere in $namespace qx$. For example:

\begin{lstlisting}

struct Employee {
	Person& person() {return person_; }
	float& salary() {return salary_; }
private:
	Person person_;
	float salary_;
};

TRAVERSABLE_OUTER(Employee, p.person_ << p.salary_)

\end{lstlisting}

If a class has no fields you should use \lstinline{TRAVERSABLE_UNIT(<type name>)} inside the 
class definition. For example:

\begin{lstlisting}

struct SubUnit {
	TRAVERSABLE_UNIT(SubUnit)
};

\end{lstlisting}

\end{comment}
If you're curious what the macroses do see Syb/Syb.h for definitions.

\subsubsection{Polymorphic types}

Polymorphic types structure are slightly harder to define. This is mostly C++ limitations.
We need at least one virtual function to dispatch needed type information. As it well known
template function can not be virtual. So besides defining shape of the data type we need to
define all generic function which will be used with the type.

First we define shape of data type. In parent class we're using macros TRAVERSABLE\_PARENT\_BEGIN(<class name>) 
instead of TRAVERSABLE\_BEGIN. If the parent is abstract use TRAVERSABLE\_ABSTRACT\_BEGIN(<class name>).
For example:

\begin{lstlisting}

struct SubUnit {
	TRAVERSABLE_PARENT_BEGIN(SubUnit)
  TRAVERSABLE_END
};

\end{lstlisting}

In all direct children types use macros TRAVERSABLE\_CHILD\_BEGIN(<class name>, <parent>) in place of 
TRAVERSABLE\_BEGIN. If a parent of a some class also has its parent use 
TRAVERSABLE\_SUB\_CHILD\_BEGIN(<class name>, <parent>, <top parent>).


For example:

\begin{lstlisting}

struct DeptUnit:SubUnit {
	Dept dept() { return dept_; }
private:
	Dept dept_;

  ACCEPTABLE_CHILD(SubUnit);
	TRAVERSABLE_CHILD_BEGIN(DeptUnit, SubUnit)
    TRAVERSABLE_FIELD(dept_)
  TRAVERSABLE_END
};

\end{lstlisting}

Next you need to make separate CPP file where all auxiliary function's templates will be insentient. 
In this file you need additionally include file ``Syb/PolySpec.h''.
And for each parent child pair put ACCEPTABLE\_INSTANCE(<parent name>, <child name>)
and ACCEPTABLE\_PARENT\_INSTANCE(<parent name>, <child name>) for each non-abstract parent.
E.g.


\begin{lstlisting}

ACCEPTABLE_PARENT(SubUnit);
ACCEPTABLE_INSTANCE(SubUnit,DeptUnit);
ACCEPTABLE_INSTANCE(SubUnit,PersonUnit);

\end{lstlisting}


See Spec.cpp in SybTest project for example.

Note \lstinline{namespace syb::qx} should be visible there.

And the unpleasant detail is necessity do define all generic functions for each data type.
It's done by means of specialisation template VisitableBy and EmptyVisitableBy (the template 
depends on kind of the functor). 
E.g.

\begin{lstlisting}

template<> struct VisitableBy<SubUnit>:Ret<mpl::vector<
			GShow
			,Generate<void>
		> >{};

\end{lstlisting}

If you forget to include some generic function here the library will not go inside
the children.

It's usual MPL function which returns MPL vector of possible generic functions for the type.
In the example it is generic show (GShow) and random data generator (Generate).

By default the template should return all possible generic functions and function's developer
is responsible for registering new in the list. So library user will only need to specialize 
the template for code optimisation. And this can be done on late stages of any project. 

It's not quite scalable but it's limitation of C++. In future we'll implement some dependency 
tracking system that will return actually needed function list for the data type. It will be
either MPL program or some external tool.

After everything is defined you can use generic function. For example to output content of 
any SYB ready object just use:

\begin{lstlisting}

  std::cout << gshow(p) << std::endl;

\end{lstlisting}

But next thing you would like to do is to change behaviour of some function for some data type.
For example you want to output double quotes around sting values. The way for doing overloading
is not standard and defined by function. You need to read function's documentation.

But for example for GShow you need to specialize template \lstinline{ShowVal}. For 
some other function it might be function overloading. Usually it's easier to manage overloading
of templates with one static function rather function parameter based overloading (like for
\lstinline{std::ostream& operator<<(std::ostream&,T const&)}.

So, for example, for GShow if you need special treatment of strings define:

\begin{lstlisting}
	
  namespace syb {
		template<>
		struct ShowVal<std::string> {
			// this is executed before for every field before traversing its children
			inline static bool before(std::ostream& out, std::string const& v) {
				out << "\"" << v << "\" ";
				// if we don't need to traverse any children return false
				// this doesn't make any sense for string, 
				// since it doesn't have children
				return false;
			};
			// this is executed after traversing children
			inline static void after(std::ostream& out, std::string const& v) {}
		};
	}

\end{lstlisting}

\subsubsection{Type aliases}

The problem will arise if we don't need to specialize all stings but only one, for example 
we want distinguish names. This can be done using type aliases. Type alias is just a C++
class that stores just one field of original type. And so you'll be able to specialize 
generic function for a given data type.

The alias can either store value or a reference. The first way is more preferable, since we
don't need to treat specially in generic functions. The aliases is defined in TypeAlias.h.

To make value based alias for name put somewhere on top level:

\begin{lstlisting}

NEWTYPE(Name,std::string);
NEWTYPE(Address,std::string);

\end{lstlisting}

And in the class where a string should be detected as Name:

\begin{lstlisting}

	struct Person {
		Person(){}
		Name& name() { return name_; }
		Address& address(){ return address_; }
	private:
		Name name_;
		Address address_;
		TRAVERSABLE(Person, p.name_ << p.address_)
	};

\end{lstlisting}

Than you can specialize aliases instead of original types. For example:

\begin{lstlisting}

	namespace syb {
		template<>
		struct ShowVal<Name> {
  		inline static bool before(std::ostream& out, Name const& v) {
        out << "{" << v.val() << "}";
				// if we return true here the library will 
        // continue traversing and execute std::string 
        // default method 
				return false;
			};
			inline static void after(std::ostream& out, Name const& v) {}
		};
	}

\end{lstlisting}

This will define class Name with the only field with type name. The type is inherited from
NewType template that is defined in TypeAlias.h. See reference for details. \todo{label}

If there is no possibility to change class members you can use reference based aliases.

For this define empty type that will be used as a tag for overloading. And a field list
definition should be changed.

For example:

\begin{lstlisting}

  // we can not change anything in this class
  struct Person {
		Person(){}
    std::string& name() { return name_; }
		std::string& address(){ return address_; }

    // the const version of accessors are also 
    // needed now
    std::string const& name() const { return name_; }
		std::string const& address() const { return address_; }
	private:
		std::string name_;
		std::string address_;
	};

  // here we provide type shape definition

  struct Name{};
  struct Address{};

  TRAVERSABLE_OUTER(Person, alias<Name>(p.name()) << alias<Name>(p.address()));

\end{lstlisting}

And such aliases are implemented using NewTypeRef template. So they can be specialized 
for example like:

\begin{lstlisting}

	namespace syb {
		template<>
		struct ShowVal<NewTypeRef<Name,std::string> > {
  		inline static bool before(std::ostream& out, 
        NewTypeRef<Name,std::string> const& v) 
      {
        out << "{" << v.val() << "}";
				// if we return true here the library will 
        // continue traversing and execute std::string 
        // default method 
				return false;
			};
			inline static void after(std::ostream& out, 
        NewTypeRef<Name,std::string> const& v) {}
		};
	}

\end{lstlisting}

It's also possible to have default value for an alias, 
use NEWTYPE\_WITH\_DEFAULT(<name>,<type>,<default value>) for this.

\begin{comment}
\subsubsection{Views}

\emph{WARNING: This is experemental implementation, and might not work in the current version}

As it was noticed before the library can be blamed for broking encapsulation principle. 
The problem become serious in case if we don't have permissions for changing classes
definition. And the class doesn't have suitable method for returning references to its
members. This is very frequent issue since it's very rare the class returns references
to its members. It's not an OOP way. You can use views for such cases.

For example:

\begin{lstlisting}
	struct Person {
		Person(){}
		Address& address(){ return address_; }
		Address const& address() const { return address_; }

		void setName(Name const& v)  { name_ = v; }
		Name const& getName() const  { return name_; }
	private:
		Name name_;
		Address address_;
	};

	TRAVERSABLE_OUTER(Person, 
		view(p
			,boost::function<Name const& (Person const&)>(&Person::getName)
			,boost::function<void (Person&,Name const&)>(&Person::setName)
			)
		<< p.address())
\end{lstlisting}

Here Name is accessed using more OOP type accessors. And so we need to create 
\emph{view} for it. See View.h for reference.

\todo{more information}

\subsubsection{Advanced usage}

The main function of the library is gfoldl. It does the main traversal job. You will not 
need to redefine it usually. But for some advanced usages just overload it for some 
parameter and do whatever you want to do. See reference of gfoldl function on doxygen
generated reference.

\todo{label}

\subsubsection{Limitation}

If you need the data type is usable for producer functions the data type should
be default constructible. Many Java framework lives this the same limitation. And
its actually can be easily removed if we find enough amount of use cases for it.

It probably worth use some template for constructing objects without providing
constructor arguments for some 3rd party library types that we are not allowed
to change. But it's not implemented yet.

\subsubsection{Resulting type definition}

See \emph{ParadiseBM.h} for working example.

\begin{lstlisting}

namespace qx {
	using namespace syb;

	NEWTYPE(Name,std::string);
	NEWTYPE(Address,std::string);

	struct Person {
		Person(){}
		Name& name() { return name_; }
		Address& address(){ return address_; }
	private:
		Name name_;
		Address address_;
		TRAVERSABLE(Person, p.name_ << p.address_)
	};

	struct Salary{};

	struct Employee {
		Person& person() {return person_; }
		float& salary() {return salary_; }
	private:
		Person person_;
		float salary_;

		TRAVERSABLE(Employee,p.person_ << alias<Salary>(p.salary_))
	};

	NEWTYPE(Manager,Employee);

	struct SubUnit {
		ACCEPTABLE(SubUnit);
		TRAVERSABLE_EMPTY(SubUnit)
	};

	struct ShowConstrs {
		template<typename T>
		void visit(Empty<T> const& v) const {
			std::cout << "Constr:" << TypeName<T>::value() << std::endl;
		}
	};

	namespace syb {
		template<> struct VisitableBy<SubUnit>:Ret<mpl::vector<
			GShow
			,GShow::Child<boost::shared_ptr<SubUnit> const>
			,GShow::Child<boost::shared_ptr<SubUnit> >
			,Generate<void>
			> >{};

		template<>
		struct EmptyVisitableBy<SubUnit>:mpl::vector<ShowConstrs>{};
	}

	struct PersonUnit:SubUnit {
		Employee& empl() { return empl_; }
		ACCEPTABLE_CHILD(SubUnit);

	private:
		Employee empl_;

		TRAVERSABLE(PersonUnit,p.empl_)
	};

	struct Dept {
		Name& name() { return name_; }
		Manager& manager() { return manager_; }
		std::list<boost::shared_ptr<SubUnit> >& subUnits() { return subUnits_; }

		Name const& name() const { return name_; }
		Manager const& manager() const { return manager_; }
		std::list<boost::shared_ptr<SubUnit> > const& subUnits() const { return subUnits_; }

	private:
		Name name_;
		Manager manager_;
		std::list<boost::shared_ptr<SubUnit> > subUnits_;
	};

	//NOTE: you will have to have const and not const options for accessors
	// the definition should be in qx namespace
	TRAVERSABLE_OUTER(Dept,p.name() << p.manager() << p.subUnits());

	struct DeptUnit:SubUnit {
		Dept dept() { return dept_; }
	private:
		Dept dept_;
		float salary_;

		ACCEPTABLE_CHILD(SubUnit);
		TRAVERSABLE(DeptUnit, p.dept_ << alias<Salary>(p.salary_));
	};

	//NOTE: encapsulation is broken

	struct Company {
		std::list<Dept>& depts() { return depts_; }
	private:
		std::list<Dept> depts_;
		TRAVERSABLE(Company,p.depts_)
	};

\end{lstlisting}

\end{comment}

\subsection{Function's catalog}

\subsubsection{Output readable value}

This is very preliminary version for the function. And it can be used now only for debug
purposes, i.e. output content of any traversable object. In future here must be done
at least punctuation handling and at most some natural language grammar handling i.e.
plural/singular forms, cases handling and so on.

There is std::ostream manipulator for easy running the function. All you need to output
an object content is run:

\begin{lstlisting}

  std::cout << gshow(c);

\end{lstlisting}

here \emph{c} is any traversable object.

To overload behaviour specialize template ShowVal:

\begin{lstlisting}

  template<typename T, typename EnableT=void>
	struct ShowVal {
		inline static bool before(std::ostream& out, T const& v);
		inline static void after(std::ostream& out, T const& v);
	};

\end{lstlisiting}

The function before is called prior to traversing children and if return value is false it 
will not traverse child further (function after also will not be called). 

The after function is called after all children are traversed.

\subsubsection{Generate random value for testing}

The function generates random values for testing purposes.
For example:

\begin{lstlisting}
	Company g;
	generate(g);
\end{lstlisting}

This will generates Company object with size 10. 

To overload the behaviour you need to specialize 

\begin{lstlisting}
		template<typename T, typename TagT, typename EnableT=void>
		struct Arbitrary{
			inline bool operator()(int s, typename CleanType<T>::type& val) const;
		};
\end{lstlisting}

Here as in previous example T is type to be specialized. TagT can be used for defining
various strategies of value generation. Meta function removes all reference aliases type
tags from a type. Current size is provided by parameter s. If size is changed during 
execution of the function the parameter's type should be reference to int.

For example:

\begin{lstlisting}

  template<typename TagT>
  struct Arbitrary<float,TagT> {
  	inline bool operator()(int s, float& val) const {
	  	val = random();
		  return false;
  	}
  };

\end{lstlisting}

As in GShow case the returns value tells children should be traversed or not.

Very often you need to return not just random value but value from some set for this 
purposes you can use OneOf helper function. For example:

\begin{lstlisting}

	namespace syb {
	template<typename TagT>
	struct Arbitrary<Name,TagT>:OneOf<Name> {
		Arbitrary() {
			reg(Name("Homer Simpson"));
			reg(Name("Marge Simpson"));
			reg(Name("Bart Simpson"));
			reg(Name("Lisa Simpson"));
			reg(Name("Maggie Simpson"));
			reg(Name("Kent Brockman"));
			reg(Name("Barney Gumble"));
			reg(Name("Krusty the Clown"));
			reg(Name("Troy McClure"));
			reg(Name("Abraham Simpson"));
			reg(Name("Seymour Skinner"));
			reg(Name("Waylon Smithers"));
			reg(Name("Charles Montgomery Burns"));
		}
	};

\end{lstlisting}

This will randomly return names with a given values.

Note NewTypeRef you should treat specially. For example:

\begin{lstlisting}

  template<typename TagT>
	struct Arbitrary<NewTypeRef<Salary,float>,TagT>
      :OneOf<NewTypeRef<Salary,float> > {
		Arbitrary() {
			reg(10);
			reg(100);
			reg(1000);
			reg(10000);
			reg(100000);
			reg(1000000);
			reg(10000000);
		}
	};

\end{lstlisting}

Here float is equal to \lstinline{CleanType<Salary>::type}.

The size is very important parameter. If you have a recursive type. For example
Dept in the current example. The generation can hangs because random number generator
can always choose recursive branch. So you need to reduce size and don't generate
recursive branch if the size parameters is less than some predefined value. For example in
the current sample type recursion can be introduced by \emph{SubUnit}. So we need
to reduce size there. And use default behaviour if the size is suitable.

\begin{lstlisting}

  template<typename TagT>
	struct Arbitrary<boost::shared_ptr<SubUnit>,TagT>:PolyArbitrary<SubUnit,TagT> {
    // first parameter is reference to int not just int as in previous examples
		inline bool operator()(int& s, boost::shared_ptr<SubUnit>& val) const {
			if (s > 8) PolyArbitrary<SubUnit,TagT>::operator()(s,val); 
			s--;
			return true;
		}
	};

\end{lstlisting}

\subsubsection{XML reading/writing}

\emph{See SybXml2}

Serialize any object to custom XML format.
Deserialize any object to custom XML format. Usually just
\begin{lstlisting}
  MyClass myobj;
  f >> syb::fromXml(myobj);
\end{lstlisting}
is enough for reading object's state. And 
\begin{lstlisting}
  MyClass myobj;
  f << syb::toXml(myobj);
\end{lstlisting}
is enough for writing.

It's possible to redefine default formating. For example, specializing 
template XmlKind with either XmlElement, XmlAttribute or XmlText types will
format corresponding type in specified way. For example:
\begin{lstlisting}

template<>
struct XmlKind<MyType>:XmlAttribute{};

\end{lstlisting}

This will read from or write to XML attribute. Though this makes sense only for aliases
of primitive types.

Also by means of XmlAttributeName and XmlElementName it's possible to change output
name, which is equal to TypeName by default. 

Type of an object can be either dispatched by attribute if NodeAttribute template
is specialized for the type or by name if it's not. For more complex dispatching it's possible
to specialize XmlElementId template.

\subsection{Dynamic functions}

For some simple generic functionality it's possible to use a dynamic function. The function
just traverses whole tree and executes a dynamic function, i.e. a function which has 
extended boost::any as it's argument, and this is responsibility of the function to 
understand its real type and use the object.

See (Syb/Any.h) for details.

There are many ways to store value of any time, for example in Java and C\# it's just any field
with \emph{Object} type, in C it's pointer to void and for C++ the best option is boost::any.

But the value has no use if we don't know its type. And here we can either check the value's type is 
equal to some type from some predefined set of types. For example with boost::any\_cast, dynamic\_cast 
or type\_info comparison (\emph(instanceof) operator in Java, \emph{is} in C\#). This method clearly
has many disadvantages, we need to know the set of possible types, and dispatching procedure consists
of many runtime checks for looking up the stored type.

On the other hand we can store a function which takes a value of stored type along with the value (and 
this is actually our extended \emph{Any} type (from Syb/Any.h) is doing. This is similar to existential type  
from functional programming world and to usual dynamic method call dispatching from OOP world. In OOP we
might store a reference to some object by its parent type. The reference might point to instances of many 
other types, i.e. children of the parent type. But during a call of any function from its interface the 
actual type of the object will be dispatched to the function's parameters (usually hidden parameter 
\emph{this}).

The same is implemented in syb::Any, but only one function can be dispatched (can be easily extended
by returning an object with required interface from the function). The function is set by means of 
\lstinline{AnyFun<Ctx,T>}, which is replacement of \lstinline{boost::function<void(CtxT&,T const&)> }. 
The replacement is required for improving build time performance and size of resulting binaries.

There's also an attempt for traversal strategies definition, but it's not yet finished.
See (Syb/Strategies.h).

\subsection{Generic diff/patch}

This function is analogues for Unix diff/patch but they work for arbitrary SYB enabled objects.
See (Syb/GenericDiff.h, Syb/GenericPatch.h, SybXml2/XmlDiffPatch.h).

\subsubsection{Command line parsing}

\emph{Not implemented}

Ability to automatically derive command line parser for any data type with documentation.

\subsubsection{XML serialization}

\emph{See SybXml2}

Serialize any object to custom XML format.

\subsubsection{XML deserialization}

\emph{See SybXml2}

Deserialize any object from custom XML format.

\subsubsection{Generate Meta Data in form of ADT}

\emph{Not implemented, use qxt instead}

Generate ADT for data types for further processing with various specific tools, what are more supposed to deal with meta data processing.
E.g. term rewriting tools, Haskell.

\subsubsection{Generate documentation for data types}

\emph{Not implemented, use qxt instead}

This will be done by means of generated ADT. It will generate \LaTeX sources for farther processing.

\subsubsection{Java Types}

\emph{Not implemented, use qxt instead}

The library is used in heterogeneous systems, there are a few different languages and technologies are used.
The library will allow automatic generation of interface datatypes on Java Side.

\subsubsection{C wrappers}

\emph{Not implemented, use qxt instead}

We need C function wrapper to be available. Since all other language such as 4GL, Java, Haskell don't support FFI for C++. 
So we need automatic facility for exporting C++ to C.

\subsubsection{4GL wrappers}

\emph{Not implemented, use qxt instead}

The wrapper function which makes function available from 4GL code.

\subsubsection{ORM}

\emph{Not implemented, use qxt instead}

The function will automatically create needed schema, and all CRUD queries for the data type.

\subsubsection{UI} 

\emph{Not implemented, use qxt instead}

This is one more possible application of the library. We can automatically generate UI widgets.

This library provides facilities for automatic creation of GUI Widget from SYB-compatible type.
The most general way to create widget is:

\lstinline{QWidget *v = new GWidget<Type>();} \footnote{The description is taken from my previous implementation of a similar library 
(what was based on Qt)}

Representation and behaviour of a widget is based on a shape of a type.
But common specializations are defined for all widgets.
There are number of templates by means of which you can change default representation or behaviour.

A parent of a generic widget is always some Widget (depends on type shape).
For example for \lstinline$std::vector$ it could be TableView or TreeView or ListView.
The librery is hightly customizable by means of C++ template specializations.

Any generic widget must have parent with type \lstinline$StylelessWidget<T,...>$.
You could respecialize this template for changing behavioral aspects of a widget.

Widget's type depends on context of a widget. For example top level widget is usually \lstinline$GWidget<T>$.
But for widget embedded into compound form (product type) has type \lstinline$ProdItem<T,...>$.
So you can specialize changes of representation and behaviour by type and context.
Almost every template has additional parameter TraiT. It helps in more accurate template specializing.
The library defines 2 traits (Editable and Readonly). They show whether widget's content can be edited by a user or not.
Your own Trait should be inherited from one of this ones.

Every generic widget has methods:

\begin{lstlisting}
T get() const;			// usual setter method
void set(T const&);		// usual getter method

and signals:
void gChanged();		// this signals value of a widget is changed
void gEdited();			// this signals value of a widget is edited by a user
\end{lstlisting}

There are number of other features.

\begin{itemize}
\item Lazy updates of widget's value without blocking of UI (from remote server for example).
\item Binding value of a container's widget with it's element widget.
\item Validators.
\item ...
\end{itemize}
\begin{comment}
\subsubsection{gmapT}
This function executes a generic function for each member of any object 
Only one direct sub-level is traversed.
The function is implemented by means of gfoldl but it's used more often.
\paragraph{gmapTCopy}
Analogues to gmapQ but it copies all objects along with traversal, so parameter object is unchanged.
\paragraph{gmapQ}

\paragraph{everywhere}
Generic function which executes generic function passed to it as parameter for each sub member of specified object recursively.
It defers from gmapT which traverses only one level.
Actually the function is analogues to familiar OOP Visitor pattern.

\paragraph{everywhereCopy}
Generic function which executes generic function passed to it as parameter for each sub member of specified object recursively.
It defers from gmapT which traverses only one level.
Actually the function is analogues to familiar OOP Visitor pattern.

\paragraph{everywhereBut}

The function is similar to everwhere but it has additional predicate which asks the library to stop traversal.

\paragraph{everything}

Generic function which executes generic function passed to it as parameter for each sub member of specified object recursively.

\paragraph{Path combinators}

Expression for accessing some node in a heterogeneous try of objects.

\end{comment}

\section{How to write generic functions}

\subsection{Generic function}

Generic function can be any C++ function object which accepts parameters of any type. 
This can be done by templated \emph{operator()} with the only argument. 


\begin{lstlisting}

  template<typename T>
  void operator()(T& v) const;

\end{lstlisting}

The function is called by \emph{gmap} during the traversal. Usually \emph{gmap} 
just applies the function to each field of provided record, and the function does
some useful actions according to type information. Haskell version of the library 
uses more generic traversal operators but for C++ it's enough since it's possible
to propagate state and return value by means of state of generic funcion's objects.


\section{Traversal strategies}

gfoldl function itself is quite hard to use. So there are a few predefined 
traverse strategies that reduce flexibility of original gfoldl but ease usage.
The first strategy is 

\begin{comment}
\subsection{Generic functions that uses void as state}

It is simplification of Spine view and it's usually used instead of the spine view.
The main change is now we assume generic function returns void. 

The main reason for doing such simplification is difficulty of maintenance return 
values in C++. But for function's state state of an object that implements the function
can be used. It's less flexible but it suits for usual use cases.

Such strategy is implemented in \emph{ReturnVoid.h} using \emph{ReturnVoid} generic function.
The implementation uses Curiously Recurring Template Pattern idiom (\todo{link}), so 
a resultless generic function should have \emph{ReturnVoid} template instance as parent.

For such functions constr method is useless so we need only analogues of apply that returns
void. During traversal library calls \emph{operator()} function for each object in type tree.
The operator should accept every types. So there should be at lest one template version
of the operator what accepts default case.

\subsection{Guarded traversal}

This is one more strategy. It allows execution of custom code before and after traversing the 
children. The strategy is implemented by \emph{Guarded} template in \emph{Traverse.h}. 

The library calls \emph{before} method before traversing object's children. And method \emph{after}
after traversing. 

Both methods have single parameter that as in all other generic functions should accept object of 
any type. If return value of \emph{before} is convertible to false the children are not traversed
and method \emph{after} is not called.

The strategy also uses Curiously Recurring Template Pattern and by default the same 
generic function object is used during the traversal. So you can keep in state of the 
object any context information.

It's often desirable to simulate implicit state stack. That's mean every changes
to state variable is visible only during traversal of one level. And then level is 
fully traversed the changes returned back to original values.

Such behaviour can be turned on by specializing template ImplicitStack to true. 
See Generate function for example. Generate function needs changes of the size 
parameter will be visible only for once concrete object and all its children. 
But not for upper level.

If the template argument is set object that implements the strategy is copied
for each level and it's discarded then the level is traversed. Of course it shell
support copying.

Global changes are still possible for such strategy. They can be modeled by storing 
references to values but not values themselves.

\subsection{Contextual traversal}

This strategy is subclass of Guarded strategy. So it also uses \emph{before} and 
\emph{after} methods. But additionally it calls \emph{context} method for each sub object.
The method has 2 parameters. The first one is parent for current object and the second
is actually current object. So this is only one level contextual traversal. 

It's not a big problem to implement arbitrary level traversal, but we need to
find out use cases for it.

Using such strategy introduces trouble for polymorphic methods. Now dependency list
for any polymorphic type should be extended with FunName::Child<Param Type>. This can
be done automatically by generic function but this is not implemented yet. For arbitrary
level traversal it's not solvable at all, so we'll have to use dynamic dispatch mechanism
for implementation if it will be required.

\subsection{Using C++ functors for making generic functions}

It's not implemented yet.

\todo{}

\end{comment}

\section{Consumers}

This is mostly policy for making generic consumer function, e.g. pretty printing, serialization, querying, transforming objects.
Consumer is usual generic function. Usually defined with Guarded traversal schema. And additionally
it makes some action.

\subsection{Example of Consumer}

See GShow for example.

\section{Producers}

This is mostly policy for making generic producer function, e.g. parsers, random data generators, deserializations.
Unlike 

\subsection{Example of Producer}

\subsection{Idioms}
\subsubsection{Curiously Recurring Template Pattern}

See \url{http://en.wikibooks.org/wiki/More_C%2B%2B_Idioms/Curiously_Recurring_Template_Pattern} for derails.

In CRTP idiom, a class T, inherits from a template that specializes on T.

\begin{lstlisting}
struct SomeClass:X<SomeClass> {...};
\end{lstlisting}

The idiom has one dangerous issue. It's wrong name of a class provided
for template's argument.

\begin{lstlisting}
struct SomeClass1:X<SomeClass2> {...};
\end{lstlisting}

There is a workaround for checking this. But it needs to be implemented.
See \url{http://www.ddj.com/cpp/184403883} for details.

\subsubsection{boost::enable\_if}

The library uses boost::enable\_if library for specializing templates.

\todo{describe main strategies.}

\subsubsection{boost::variant}

Boost::variant is example of Sum type.
In OOP Sum type is Abstract class this some number of children, each is a different element of a Sum.
Although typical to OOP types is supported many boilerplate code programmer should write by himself.
But with for \lstinline$boost::variant$ great amount of this routine boilerplate job is done by this library.

\todo{this is not implemented yet}

\subsubsection{Pimpl}

The idiom is very important because heavy use of templates generates many instances of the same 
code that will lead to huge binary sizes and probably will break compilation. 

So content of function should simply dispatch parameters to non-template functions that are instantiated
in usual for C++ way (i.e. in cpp file). This way template functions will be small and they very probably
will be inlined and so quality of resulting code will be near to hand written, but without mistakes intrinsic
to hand written code.

\section{Future work}

See ToDo list from generated by doxygen reference documentation.

\end{document}

